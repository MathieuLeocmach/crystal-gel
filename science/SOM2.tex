%\documentclass[12pt]{article}
\documentclass[12pt,onecolumn]{revtex4}
%\documentstyle[aps,multicol,eincludegraphics]{revtex}
%\documentclass[twocolumn]{revtex4}
\usepackage{graphicx}
%\usepackage{scicite}
% \usepackage{epstopdf}
\usepackage{amsmath}
%\usepackage{times}

\topmargin 0.0cm
\oddsidemargin 0.2cm
\textwidth 16cm 
\textheight 21cm
\footskip 1.0cm




\begin{document}
\title{``Colloidal cold rain'' as a novel route to crystallization via viscoelastic phase separation} 

\author{Hideyo Tsurusawa$1$, John Russo$^1$, Mathieu Leocmach$^{2}$ and Hajime Tanaka$^{1,\ast}$ 
\\
\normalsize{$^1$Institute of Industrial Science, University of Tokyo,}\\
\normalsize{4-6-1 Komaba, Meguro-ku, Tokyo 153-8505, Japan}\\
\normalsize{$^2$Institut Lumière Matière, CNRS UMR 5306, Université Claude Bernard Lyon 1, }\\
\normalsize{Université de Lyon, Lyon, 69622 Villeurbanne Cedex, France}\\
\normalsize{$^\ast$Corresponding author. E-mail: tanaka@iis.u-tokyo.ac.jp}}

%\date{Received August 11, 2011}
\maketitle

%\centerline{\bf \Large Supporting Online Material}


\noindent
{\bf METHODS}

\noindent
{\bf Experimental.}
We use \textsc{pmma} (poly(methyl methacrylate)) colloids sterically stabilized with methacryloxypropyl terminated \textsc{pdms}(poly(dimethyl siloxane)) and fluorescently labelled with rhodamine isothiocyanate chemically bonded to the \textsc{pmma}. 
The average colloid diameter is 2.1 $\mu$m. 
We assess that the size distribution of our particle is Gaussian with a polydispersity below 5\% via direct confocal measurements~\cite{Leocmach2013}.
This small polydispersity allows crystallization.
We disperse the particles in refractive index and density matching mixture of cis-decalin (Tokyo Kasei) and bromocyclohexane (Sigma-Aldrich).

To induce short-ranged depletion attraction, we use polystyrene (TOSOH) of molecular weight 3.8 Mdalton as non-adsorbing polymer. 
Experiments are conducted at 27 $^\circ$C, some 80 $^\circ$C above the theta temperature in this solvents mixture~\cite{Royall2007}. A Flory scaling of the measurements of~\cite{lu2008gelation} yields a radius of gyration $R_g=76$ nm, and thus the polymer-colloid size ratio is $q_R=2R_g/\sigma=0.07$.

In the absence of salt, the Debye length is expected to reach several $\mu$m and the (weakly) charged colloids experience a long range electrostatic repulsion. We confirm that colloids never come close enough to feel the short-ranged attraction. Screening by tetrabutylammonium bromide (Fluka) at saturated concentration brings down the Debye length to about 100 nm, practically discarding the repulsion. 
Thus, salt injection can screen the Coulomb repulsion and make the polymer-induced depletion attraction effective, initiating phase separation. 

Our sample cell has two layer compartments separated by membrane filters. The first layer is a sample container with thickness of 200 $\mu$m. A mixture of colloids, polymer, and solvent without salt is  set in the first layer for microscopy observation. The second layer is a salt-reservoir with a half-opened structure, which allos us to exchange or insert a reservoir solution. The volume ratio of the first and second compartments is approximately 1:100. These two compartments are separated by a membrane filter with its pore size of 0.1 $\mu$m, which passes only salt ions. 

In our experiments, the reservoir solution was initially a polymer solution without salt and electrostatic repulsion by the unscreened surface charges inhibited colloidal aggregation. Under microscopy observation, we quickly exchanged the reservoir solution into a polymer solution with salt at saturated concentration (4 mM) and sealed the half-opened reservoir with cover glass to avoid evaporation of solvent. Salt diffused into the first layer within a few minutes, typically 2 minutes, screened the surface charges, and initiated colloidal aggregation. 

The data are collected on a Leica SP5 confocal microscope, using 532 nm laser excitation. The scanning volume is 98 $\times$ 98 $\times$ 53 $\mu$m$^3$, which contains $\sim 10^4$ colloid particles.  The Brownian time is $\tau_B \approx 4.5$ s. To be able to follow individual trajectories, we perform a 3D scan every 10 s at early time and every 30 s later.





%\paragraph*{State points studied}
%The experimental data is taken at two different volume fractions ($\phi=0.10$ and $\phi=0.25$) and for different values
%of the polymer concentration, $c_p=0.38,0.48,0.57,1.07$ for $\phi=0.25$ and $c_p=0.82,1.36$ for $\phi=0.10$.
%For each state points, scans at early times (every $10s$) and at the late stage (every $30s$) of the gelation process.
%
%We will consider two particles bonded if their center-to-center distance is within $12$ pixels. The radius of a colloidal
%particle is $5.365$ pixels.

\noindent
{\bf Structural analysis.}
To account for imprecision in particle localisation close to contact, we consider two particles bonded if their center-to-center distance is within 2.3 $\mu$m $>\sigma+2R_g$.

To detect percolation, two different methods have been employed. In the first one we detect percolation
by looking at which frame the cluster size distribution has a more extended power-law decay. In the second technique,
we simply measure the spatial extent of the largest cluster and consider it percolating when it is comparable to the
size of the field of view of the microscope. Both methods lead to essentially identical percolation time for each state point.





%\bibliographystyle{Science}
\bibliographystyle{apsrev}
\bibliography{biblio}



\end{document}
