\documentclass[11pt]{article}
\usepackage{graphicx}
\usepackage{amssymb}
\usepackage{epstopdf}
\usepackage{color}
\usepackage{setspace}

\begin{document}

\doublespacing

\noindent
{\bf Dear Editor,}
\vskip 0.3cm
Please find enclosed in the present submission our manuscript entitled
\emph{``Colloidal rain'' as a novel route to crystallization via viscoelastic phase separation},
that we would like you to consider for publication in \emph{Science}.

Our Report represents the first particle-level study of crystal growth via the Wegener-Bergeron-Findeisen mechanism.
The process occurs naturally in mixed phase clouds, between 0$^\circ$ C and -38$^\circ$ C, where supercooled water and ice crystals coexist,
and is responsible for the growth of ice crystals in clouds and rain formation. But we argue that this
mechanism of crystal growth is much more common than previously thought,
and show that it can be accessed directly in a Soft Matter model system.
We study a mixture of poly(methyl methacrylate) colloids with non-adorbing polymer (polystyrene)
with confocal microscopy experiments, that give us access to the full kinetic process with
particle resolution. We have built a novel experimental setup that allows the evolution of
the system to be observed directly from the very early stages and in absence of spurious fluid flows.
This allows us to directly study the dynamics of phase separation and of crystal growth at single-particle resolutions,
and for the first time characterize the Bergeron process at a microscopic level.

Understanding the process of crystal growth in mixed-phase systems is of fundamental importance,
and has several technological applications. As an example, understanding
water crystallization in clouds is recognized as one of the main challenges in the
modeling of Earth's radiation budget and climate. \textcolor{red}{Moreover, our study addresses the formation of crystal networks,
which are observed in many systems, including magma, biominerals and foods, where the origin
and formation of the networks still lacks fundamental understanding.}
The difficulties arise both from our poor understanding of small-scale microphysical effects,
and to the difficulty of experimental investigations. In our colloidal system, both
difficulties are overcome due to the colloid's accessible length and time scales.

Our manuscript thus represents a fundamental investigation on one of the most important
crystal-growth mechanisms, that will resonate with the broad Science community.
Moreover, it highlights a very deep connection between the disciplines of Soft Matter and Climate Modeling,
and we thus believe that will have broad appeal to the readership of Science.

We hope that you will consider it for publication.



\vskip 0.3cm
For your convenience, we provide here a list of possible Referees who are widely considered to be experts in the field.
\begin{itemize}
\item Daniel Bonn, University of Amsterdam, D.Bonn@uva.nl
\item Prof. Henk Lekkerkerker, University of Utrecht, H.N.W.Lekkerkerker@uu.nl
\item Prof. Francesco Sciortino, University of Rome ``La Sapienza'', \\ francesco.sciortino@phys.uniroma1.it
\item Prof. Michael Solomon, University of Michigan, mjsolo@umich.edu
\item Prof. David Weitz, Harvard University, weitz­@seas.harvard.edu
\end{itemize}






\vskip 0.3cm
\noindent
Warm thanks to you for your kind attention.

\vskip 0.8cm

{\bf 
\noindent
Sincerely yours,
\vskip 0.3cm
\noindent
Hideyo Tsurusawa, John Russo, Mathieu Leocmach, \\ 
and Hajime Tanaka
}

\end{document}

