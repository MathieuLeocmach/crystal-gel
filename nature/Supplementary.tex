%\documentclass[12pt]{article}
\documentclass[12pt,onecolumn]{revtex4-1}
%\documentstyle[aps,multicol,eincludegraphics]{revtex}
%\documentclass[twocolumn]{revtex4}
\usepackage{graphicx}
%\usepackage{scicite}
% \usepackage{epstopdf}
\usepackage{amsmath}
%\usepackage{times}

\topmargin 0.0cm
\oddsidemargin 0.2cm
\textwidth 16cm 
\textheight 21cm
\footskip 1.0cm




\begin{document}


\noindent
{\bf Supplementary Information}

\vspace{5mm}
\noindent
\emph{\bf Supplementary Movie 1}. This movie shows the early stage of gelation process of $\phi\approx 0.30$ and $c_p=0.38$ mg/g, from the original confocal images at the same z-position as Fig.~3a. Total duration is 1160 s (530~$\tau$). We quickly started scanning the sample after exchanging the salt-reservoir with the solvent including salt. At the beginning, salt had not reached the scanning volume and colloids were dispersed by electrostatic repulsion. At 0:02 in the movie, salt diffused into the scanning volume and screened the surface charges of colloids. Then, colloids immediately initiated phase separation by short-ranged attractions.

\vspace{5mm}
\noindent
\emph{\bf Supplementary Movie 2}.
This movie shows the late stage after Supplemental Movie 1 with scanning duration of 27000~s (12300~$\tau$).


\end{document}
