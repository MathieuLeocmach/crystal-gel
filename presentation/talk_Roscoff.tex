\documentclass[xcolor=table]{beamer}
\usepackage[utf8]{inputenc}
\usepackage[british]{babel}
\usepackage[super]{nth}
%\usetheme{Boadilla}
%\usecolortheme{rose}
%\usecolortheme{crane}
%\usefonttheme{structuresmallcapsserif}
%\setbeamertemplate{navigation symbols}{}

\definecolor{Main}{rgb}{0.74, 0.13, 0.19}
\definecolor{Accent1}{rgb}{0.76,0.36,0.13}
\definecolor{Accent2}{rgb}{0.54,0.1,0.4}

\mode<presentation>{\usetheme{ilm}}
%\usecolortheme{rose}
%\useinnertheme[shadow]{circles}
%\usecolortheme{whale}
%\useoutertheme{infolines}

%\usecolortheme[named=Accent1]{structure}




%\setbeamerfont{page number in head/foot}{size=\large}
%\setbeamercolor{page number in head/foot}{fg=Main}
%% page/total
%%\setbeamertemplate{footline}[frame number]
%% pas de total
%\setbeamertemplate{footline}{%
%    	\hfill%
%	\usebeamercolor[fg]{page number in head/foot}%
%	\usebeamerfont{page number in head/foot}%
%	\insertframenumber\kern1em\vskip2pt%
%}
%
%\setbeamersize{text margin left=1em}
%\setbeamersize{text margin right=1em}

\usepackage[overlay,absolute]{textpos}
\setlength{\TPHorizModule}{10mm}
\setlength{\TPVertModule}{\TPHorizModule}
\textblockorigin{10mm}{10mm} % start everything near the top-left corner
\setlength{\parindent}{0pt}

%font
\usepackage[T1]{fontenc}
\usepackage{times}
%\usepackage[oldstylenums]{kpfonts}

%\include{macros}


%proper math and math symbols
%\usepackage{amsmath}
\usepackage{amssymb}

\usepackage{datenumber,fp}

\usepackage{siunitx}

\usepackage{tabu}
\usepackage{multirow}
\usepackage{booktabs}

% Allow the usage of graphics (.jpg, .png, etc.) in the document
\usepackage{graphicx}
\usepackage{tikz}
\usetikzlibrary{arrows,shapes,backgrounds, calc, positioning, topaths,chains, intersections, decorations.markings, decorations.text, shapes.geometric, matrix,patterns,mindmap}
%\usetikzlibrary{positioning, patterns,topaths,chains,matrix}

\usepackage{pgfplots}
\usepackage{pgfplotstable}
\pgfplotsset{compat=1.9}
\usepgfplotslibrary{groupplots}
\usepgfplotslibrary{external}
\makeatletter
\newcommand*{\overlaynumber}{\number\beamer@slideinframe}
\tikzset{
  beamer externalizing/.style={%
    execute at end picture={%
      \tikzifexternalizing{%
        \ifbeamer@anotherslide
        \pgfexternalstorecommand{\string\global\string\beamer@anotherslidetrue}%
        \fi
      }{}%
    }%
  },
  external/optimize=false
}
\let\orig@tikzsetnextfilename=\tikzsetnextfilename
\renewcommand\tikzsetnextfilename[1]{\orig@tikzsetnextfilename{#1-\overlaynumber}}
\makeatother

\tikzset{every picture/.style={beamer externalizing}}
\tikzexternalize
\tikzsetexternalprefix{fig_presentation/}
%\tikzset{external/optimize=false}
%\tikzset{external/force remake}


%link or play movies
\usepackage{multimedia}



%beamer related package

\usepackage{todonotes}
\presetkeys{todonotes}{inline}{}


%bibliography
\usepackage[style=authoryear-comp, language=british,eprint=false, url=false, doi=false, sortcites=true, sorting=none, isbn=false, firstinits=true,maxcitenames=6]{biblatex}
%minimal citations
\AtEveryCitekey{%
	\clearfield{title}
	\clearfield{pages}
	\clearfield{volume}
	\clearfield{number}
	\clearfield{month}}
\newcommand{\myfullcite}[1]{{\scriptsize\fullcite{#1}}}
\renewbibmacro{in:}{%
  \ifentrytype{article}{}{%
  \printtext{\bibstring{in}\intitlepunct}}}
%\bibliography{biblio}


\newcolumntype{P}[1]{>{\raggedright}p{#1}}

\institute[iLM]{iLM, Univ. Claude Bernard Lyon 1 and CNRS, France}
\title[crystal gel]{A novel route to the spontaneous formation of porous crystals via viscoelastic phase separation}
\author[M. Leocmach]{Mathieu Leocmach}
\date{30 June 2016}
\titlegraphic{
	\begin{tabu}{X[c]X[c]X[c]}
		\includegraphics[height=3\baselineskip]{Tsurusawa}&
		\includegraphics[height=3\baselineskip]{John}&
		\includegraphics[height=3\baselineskip]{Tanaka}\\
		Hideyo Tsurusawa & John Russo & Hajime Tanaka\\
		U. Tokyo & U. Bristol & U. Tokyo\\
	\end{tabu}
	
	%\vfill
	%\includegraphics[height=2\baselineskip]{logo_ens-lyon}\quad
	%\includegraphics[height=2\baselineskip]{logo_ums_grand}\quad
	%\includegraphics[height=2\baselineskip]{../../Yaourt/CNRSfilaire-Q}\quad
	%\includegraphics[height=2\baselineskip]{CRPP}\quad
	%\includegraphics[height=2\baselineskip,clip=true, trim=6mm 14mm 6mm 0]{NEW-Logo-ERC-OUTLINE}
	}


\newlength{\mylength}

%\includeonly{creep_beamer}

\begin{document}
\tikzset{every mark/.append style={scale=0.8}}
\pgfplotsset{every axis/.append style={footnotesize}}

\pgfplotscreateplotcyclelist{earthy}{%
{red!40!black,mark=o},
{red!60!black,mark=triangle, every mark/.append style={rotate=180}},
{red!80!black,mark=square},
{red,mark=triangle},
{red!80!yellow, mark=diamond},
red!60!yellow,
red!40!yellow,
}

\AtBeginSection[]{
	\addtocounter{framenumber}{-1}
	\begin{frame}[plain]
		\tableofcontents[currentsection, hideothersubsections]
	\end{frame}
}

\begin{frame}{\pgfuseimage{cnrs-logo}\hspace*{0.3cm}\pgfuseimage{ucbl-logo}}%[plain]
	\titlepage
\end{frame}

\setcounter{framenumber}{0}

\begin{frame}{Crystallisation}
	\tikzsetnextfilename{whyx}%
	\begin{tikzpicture}
\node[text width=0.25\textwidth, anchor=base west] (process){
Crystallisation
\begin{itemize}
\item nucleation
\item growth
\end{itemize}};

\node[anchor=base east] (prop) at (\textwidth,0){
Properties
};

\node[text width=0.39\textwidth, anchor=base] (x) at ($(process.base east)!0.5!(prop.base west)$){
Crystalites
\begin{itemize}
\item polymorph type
\item shape
\item spatial arrangement
\item size distribution
\end{itemize}};


% 
\node[above left=3em and 0 of prop.east] (np){natural processes};
\node[below left=8em and 0 of prop.east] (tech){technological applications};

\begin{scope}[->,ultra thick, ilmcolor]
\draw (process.base east) +(0,0.3em) -- ($(x.base west)+(0,0.3em)$);
\draw (x.base) +(0,0.3em) -- ($(prop.base west)+(0,0.3em)$);
\draw (prop) -- (np.south-|prop);
\draw (prop) -- (tech.north-|prop);
\end{scope}
\end{tikzpicture}
	
	\begin{block}{Nucleation}
	\begin{itemize}
		\item 10-1000 molecules $\rightarrow$ difficult to observe
		\item classical nucleation theory (CNT): fluid $\xrightarrow{\text{1 step}}$ crystal
	\end{itemize}
	\end{block}
\end{frame}
	
\begin{frame}{Let it snow!}
	\hfill Wegener-\alert{Bergeron}-Findeisen \alert{process}
	\begin{columns}[b]
	\column{0.4\textwidth}
	\includegraphics[width=\textwidth,clip=true,trim=0 0 0 1cm]{Pitter1.jpg}
	
	\vspace{1em}\includegraphics[width=\textwidth]{bergeron.jpg}
	\column{0.40\textwidth}
	\includegraphics[width=\textwidth,clip=true,trim=0 0 0.8mm 0]{queries-figure-1.jpg}
	
	\vspace{1em}\includegraphics[width=\textwidth]{OKC-Fallstreak.jpg}
	\end{columns}
\end{frame}


\begin{frame}{Colloid + Polymer}
	
\begin{columns}
	\column{0.45\textwidth}
	\tikzsetnextfilename{interactions}%
	\begin{tikzpicture}
		\let\mrad\relax
		\newlength\mrad
		\setlength{\mrad}{1em}
		
		
		%schemas interactions
		
		%repulsion
		\begin{scope}[xshift=2\mrad]
		\node[circle, inner sep=0, minimum width=4\mrad, inner color=Main, outer color=white]  (puri) {}; 
		\foreach \x in {0,30,...,330}{
			\draw[decoration={random steps,segment length=0.1\mrad, amplitude=0.08\mrad},decorate] (0,0) -- (\x:1\mrad);
		}
		\fill[gray, radius=0.8\mrad] circle;
		\foreach \x in {15,75,...,315}{
			\node[Accent2, font=\tiny] at (\x:0.6\mrad) {+};
		}
		\end{scope}
		
		%spheres collantes
		\begin{scope}[xshift=\textwidth-2.5\mrad]
		\node[draw,circle, inner sep=0, minimum width=2.5\mrad, green!50!black, dashed]  (colli) {}; 
		\draw[green!50!black, radius=1.25\mrad, dashed] (1.6\mrad,0) circle;
		\fill[gray, radius=0.8\mrad] circle;
		\fill[gray, radius=0.8\mrad] (1.6\mrad,0) circle;
		\end{scope}
		
		\path (puri) -- (colli) coordinate[midway] (m);
		
		%spheres dures
		\begin{scope}[shift=(m)]
		\node[circle, inner sep=0, minimum width=2.2\mrad, inner color=Main, outer color=white]  (duri) {}; 
		\foreach \x in {0,30,...,330}{
			\draw[decoration={random steps,segment length=0.1\mrad, amplitude=0.08\mrad},decorate] (0,0) -- (\x:1\mrad);
		}
		\fill[gray, radius=0.8\mrad] circle;
		\foreach \x in {15,75,...,315}{
			\node[Accent2, font=\tiny] at (\x:0.6\mrad) {+};
		}
		\end{scope}
		
		%fleches
		\draw[->] (puri) -- (duri) node[midway,above,font=\footnotesize]{ions};
		\draw[->] (duri) -- (colli)node[midway, above,font=\footnotesize]{PS};
		% \draw[->] (puri) -- (ps) -- (depli);
		
		
		%titres
		\node[text width=6em,anchor=base west, inner xsep=0] (pur) at (puri.west|-puri.north) {repulsion};
		
		\node[text width=6em,anchor=base, align=center] (dur) at (duri|-pur.base) {hard};
		
		\node[text width=6em, anchor=base east, inner xsep=0,xshift=1.5\mrad, align=right] (coll) at (colli.east|-pur.base) {depletion};
	\end{tikzpicture}\\

	\bigskip
	\includegraphics[width=\textwidth]{../../CNRS-2015/Renth_sedim.jpg}\\
	\textit{\footnotesize Poon PRL 1999, Renth PRE 2001}
	\begin{itemize}
	\item triple coexistence
	\item various kinetic paths
	\end{itemize}

	\column{0.45\textwidth}
	\includegraphics[width=\textwidth,clip=true,trim=0 0 0 4.5cm]{Renth_diag.pdf}
	\end{columns}
\end{frame}


\begin{frame}{Another class of gel? Crystal-gel?}
	\begin{itemize}
	\item Phase separation arrested by crystallisation
	\begin{description}
		\item[polymer blend] Tanaka \& Nishi \textit{PRL} 1985
		\item[simulations] Soga \textit{J.Chem. Phys.} 1999; Fortini \textit{PRE} 2008 
	\end{description}
	\item Spinodal then crystallisation
	\begin{description}
		\item[simulations] Pérez \textit{Langmuir} 2011
		\item[microgravity] (triple coexistence) Sabin \textit{PRL} 2012
	\end{description}
	\end{itemize}
	\includegraphics[width=\textwidth]{micrograv_Sabin2012}
\end{frame}

\begin{frame}{Experimental}
How to do that in 3D at particle level \alert{without a space station}?
\begin{itemize}
\item index matched + \alert{\SI{2.3}{\micro\metre} colloids} + confocal
\item density match + $T$-control
\item preparation protocol without flow
\end{itemize}
\bigskip
\begin{columns}
	\column{0.265\textwidth}
		Colloids+polymers\\
		{\footnotesize no screening
		\begin{itemize}
			\item Dispersed
			\item Homogeneous
		\end{itemize}
		\bigskip
		\begin{beamercolorbox}[rounded=true]{block body}PMMA + PS\\
		in decalin + CHB $\epsilon_r\approx5$\\
		$\kappa^{-1}>\SI{10}{\micro\metre}$
		\end{beamercolorbox}}
	\column{0.4\textwidth}
		\tikzsetnextfilename{chemin_gelification}%
		
\begin{tikzpicture}
\let\mrad\relax
\newlength\mrad
\setlength{\mrad}{1em}
%depletition et repulsion
\begin{scope}[yshift=3\mrad]
\node[circle, inner sep=0, minimum width=4\mrad, inner color=Main, outer color=white]  (depli) {}; 
\draw[green!50!black, radius=1.25\mrad, dashed] circle; 
%\draw[green!50!black, radius=0.3\mrad] (1.1\mrad,0) circle; 
\foreach \x in {0,30,...,330}{
	\draw[decoration={random steps,segment length=0.1\mrad, amplitude=0.08\mrad},decorate] (0,0) -- (\x:1\mrad);
}
\fill[gray, radius=0.8\mrad] circle;
\foreach \x in {15,75,...,315}{
	\node[Accent2, font=\tiny] at (\x:0.6\mrad) {+};
}
\end{scope}
\begin{scope}[yshift=-3\mrad]
\node[circle, inner sep=0, minimum width=4\mrad, inner color=Main, outer color=white]  (depli2) {}; 
\draw[green!50!black, radius=1.25\mrad, dashed] circle; 
%\draw[green!50!black, radius=0.3\mrad] (1.1\mrad,0) circle; 
\foreach \x in {0,30,...,330}{
	\draw[decoration={random steps,segment length=0.1\mrad, amplitude=0.08\mrad},decorate] (0,0) -- (\x:1\mrad);
}
\fill[gray, radius=0.8\mrad] circle;
\foreach \x in {15,75,...,315}{
	\node[Accent2, font=\tiny] at (\x:0.6\mrad) {+};
}
\end{scope}

%spheres collantes
\begin{scope}[xshift=\textwidth-3.5\mrad]
\node[draw,circle, inner sep=0, minimum width=2.5\mrad, green!50!black, dashed]  (colli) at (0,0.8\mrad) {}; 
\draw[green!50!black, radius=1.25\mrad, dashed] (0,-0.8\mrad) circle;
\fill[gray, radius=0.8\mrad] (0,0.8\mrad) circle;
\fill[gray, radius=0.8\mrad] (0,-0.8\mrad) circle;
\end{scope}

\draw[->,blue!20, line width=0.1\mrad] (2\mrad,0) -- (\textwidth-5\mrad,0) node[midway,above, font=\footnotesize, blue]{ion diffusion};
\end{tikzpicture}

	\column{0.335\textwidth}
		\tikzsetnextfilename{cell}%
		\let\mlen\relax%
\newlength\mlen%
\setlength{\mlen}{24\baselineskip}%
\begin{tikzpicture}
		%box
		\draw[every node/.style={draw, inner sep=0, minimum width=0.01\mlen, minimum height=0.12\mlen, anchor=south west}] 
			node (Rrightwall) at (0, 0) {}
			node (Rleftwall) at (0.2\mlen, 0) {}
			[every node/.append style={minimum height=0.06\mlen}]
			node (Crightwall) at ($(Rrightwall.north west)+(0, 0.01\mlen)$) {}
			node (Cleftwall) at (Crightwall.south -| Rleftwall.west) {}
			(Crightwall.north west) +(0,0.005\mlen) rectangle (Cleftwall.north east);
			
	
		%filtre
		\foreach \x / \y in {0/0.23, 0.27/0.48, 0.52/0.73, 0.77/1}
			\fill[gray] ($(Rrightwall.north west)!\x!(Rleftwall.north east)$) rectangle ($(Crightwall.south west)!\y!(Cleftwall.south east)$);
		
		
		%laser
		\node[fill, green, semitransparent, isosceles triangle, anchor=apex, shape border rotate=-90, minimum width=0.08\mlen, inner sep=0,  isosceles triangle apex angle=75] at ($(Crightwall.north east)!0.4!(Cleftwall.south west)$) (laser) {};
		
		%lens
		\filldraw[lightgray] (laser.right corner) arc[start angle=180,delta angle=180,x radius=0.04\mlen, y radius=0.01\mlen] --cycle;
		%colloids
		\begin{scope}[radius=0.008\mlen, ball color=red!80,shift={(Crightwall.south west)}, yshift=-0.13\mlen]
			\shade(0.08\mlen, 0.175\mlen) circle;
			\shade(0.11\mlen, 0.16\mlen) circle;
			\shade(0.05\mlen, 0.15\mlen) circle;
			\shade(0.03\mlen, 0.18\mlen) circle;
			\shade(0.14\mlen, 0.14\mlen) circle;
			\shade(0.16\mlen, 0.165\mlen) circle;
		\end{scope}
		
		%polymers
		\begin{scope}[shift={(Crightwall.south west)}, yshift=-0.13\mlen]
		\foreach \x / \y in {0.06/0.1745, 0.17/0.1345, 0.172/0.14, 0.145/0.16, 0.19/0.18, 0.09/0.14, 0.03/0.15} 
			\draw[%
			gray, decoration={coil, segment length=0.0025\mlen, amplitude=0.0025\mlen},
			] (\x\mlen, \y\mlen)
			decorate{++(-0.0025\mlen,-0.0025\mlen) -- ++(0.0045\mlen,0.0045\mlen) -- ++(0,-0.0045\mlen) -- ++(-0.005\mlen,+0.005\mlen) };
		\end{scope}
		
		%salt
		\begin{axis}[
			at={($(Rrightwall.south east)+(0.75,0.75)$)},
			width=0.19\mlen-2*0.75,
			height=0.115\mlen,
			axis lines=none,
			scale only axis,
			xmin=-1, xmax=1,ymin=0,ymax=1,
			]
			\addplot [blue, only marks, mark=*, samples=1000, mark size=0.75]
    {rand^2};
		\end{axis}
		
		%arrows
		\foreach \x in {0.25,0.5,0.75}
			\draw[blue!20,->, line width=0.005\mlen] ++($(Rrightwall.south west)!\x!(Rleftwall.south east)$) +(0, 0.08\mlen) -- +(0, 0.11\mlen);
		
		
		%labels
		\begin{scope}[right, every node/.style={font=\footnotesize}]
			\node[anchor=north] at ($(Rleftwall.south)!0.5!(Rrightwall.south)$) (Reservoir) {Salt reservoir};
			\node[anchor=west] at (laser.left corner) {Objective lens};
			%\node[text width=0.15\mlen, align=right] at (Cleftwall-|Reservoir.west) (oc) {Observation cell};
			\node[right = 0.01\mlen of Rleftwall.north west] {Filter};
			%\node[above right,blue] at (Rleftwall.east-|Reservoir.west) {Ion diffusion};
			%\node[above right] at (0.21\mlen, 0) {Reservoir};
		\end{scope}
\end{tikzpicture}
\end{columns}
\end{frame}

\begin{frame}{Exploring phase diagram}
	\vspace{\baselineskip}
	\tikzsetnextfilename{phdiag_cluster}%
	\begin{tikzpicture}[
		every axis/.style={
		xlabel absolute, every axis x label/.append style={anchor=base, yshift=-1em},
		%ylabel absolute, every axis y label/.append style={anchor=base, yshift=-0.5em}
		}]
	\pgfplotscreateplotcyclelist{samples}{%
	blue,mark=square*,\\%
	cyan,mark=square\\%
	red, thick, mark=*\\%
	red!65!yellow, mark=otimes\\%
	red!30!yellow, mark=oplus\\%
	red!65!black, mark=o\\%
	}%
	\begin{groupplot}[
		group style={
				group name=g, group size=2 by 2,
				horizontal sep=4em,
				vertical sep=3em,
				%x descriptions at=edge bottom,
			},
		cycle list name=samples,
		width=0.51\textwidth,
		height=0.5\linewidth,
		axis on top,
	]
	\nextgroupplot[
		title={phase diagram},
		xlabel={$\phi$}, ylabel={$c_p$ (\si{\milli\gram/\gram})},
		xmin=0,xmax=0.74,
		ymin=0, ymax=1.9, ytick={0,0.5,...,2},
		mark options={solid}
		]
		\begin{scope}[dotted, mark repeat=2,mark phase=2,<->]
		\addplot file {/data/Documents/tex/svn/crystal-gel/sample0.phd};
		\addplot file {/data/Documents/tex/svn/crystal-gel/sample1.phd};
		\addplot file {/data/Documents/tex/svn/crystal-gel/sample2.phd};
		\addplot file {/data/Documents/tex/svn/crystal-gel/sample3.phd};
		\addplot file {/data/Documents/tex/svn/crystal-gel/sample4.phd};
		\addplot file {/data/Documents/tex/svn/crystal-gel/sample5.phd};
		\end{scope}
		\addplot[black, dashed,no marks] file {/data/Documents/tex/svn/crystal-gel/gasliquid_sg.phd};% node[diamond, fill, inner sep=2pt] at (current plot begin){};
		\addplot[black, dashed,no marks] file {/data/Documents/tex/svn/crystal-gel/gasliquid_sl.phd};
		\addplot[black, no marks] file {/data/Documents/tex/svn/crystal-gel/gasliquid_bl.phd};
		\addplot[black, no marks] file {/data/Documents/tex/svn/crystal-gel/gasliquid_bg.phd};
		\addplot[fill=lightgray!50, no marks, draw=none] file {/data/Documents/tex/svn/crystal-gel/fluidcrystal_f.phd} node[above right] at (rel axis cs:0,0) {Fluid} -- (rel axis cs:0,0) \closedcycle;
		\addplot[fill=lightgray!50, no marks, draw=none] file {/data/Documents/tex/svn/crystal-gel/fluidcrystal_x.phd} node[above left] (x) at (rel axis cs:1,0.5) {Crystal}  -| (rel axis cs:1,0) \closedcycle;
		%\draw[gray] (axis cs:0.371290311343, 0.12299370925) -- (axis cs:0.680497991395,0.0372636873433);
		%\draw[gray, dotted] (axis cs:3.40152108798e-05,0.624739653238) -- (axis cs:0.722831684246, 0.0726528862772);
		\draw[->] (x.base) --(rel axis cs:0.98,0.01);
		\addplot[only marks, mark=triangle] file{/data/Documents/tex/svn/crystal-gel/fluid_samples.phd};
	
	\nextgroupplot[
		title={cluster size distribution},
		xlabel={$s$}, ylabel={$R_g/\sigma$},
		xmin=1, xmax=4e3,
		ymin=0.5, ymax=30,
		xmode=log, ymode=log, 
		clip mode=individual,
		]
		\addplot file{/data/Documents/tex/svn/crystal-gel/data/fig1b_sqrt_1.dat};
		\addplot file{/data/Documents/tex/svn/crystal-gel/data/fig1b_sqrt_2.dat};
		\addplot file{/data/Documents/tex/svn/crystal-gel/data/fig1b_sqrt_3.dat};
		\addplot file{/data/Documents/tex/svn/crystal-gel/data/fig1b_sqrt_4.dat};
		\addplot file{/data/Documents/tex/svn/crystal-gel/data/fig1b_sqrt_5.dat};
		\addplot[gray, ultra thick, domain=1:4e3] {0.629*x^(1/2.53)} node [left] at (rel axis cs:1,0.9){$D=2.53$};
	
	
	\end{groupplot}
	%\draw(g c1r1.outer south west) -- +(\textwidth,0);
	\end{tikzpicture}
	\begin{itemize}
	\item Good agreement with theoretical phase diagram
	\item At percolation time, compatible with random gelation
	\end{itemize}
\end{frame}

\begin{frame}{Outlier at later times}
	\vspace{\baselineskip}
	\tikzsetnextfilename{Sq_localphi}%
	\begin{tikzpicture}[
		every axis/.style={
		xlabel absolute, every axis x label/.append style={anchor=base, yshift=-1em},
		%ylabel absolute, every axis y label/.append style={anchor=base, yshift=-0.5em}
		}]
	\pgfplotscreateplotcyclelist{samples}{%
	blue,mark=square*,\\%
	cyan,mark=square\\%
	red, thick, mark=*\\%
	red!65!yellow, mark=otimes\\%
	red!30!yellow, mark=oplus\\%
	red!65!black, mark=o\\%
	}%
	\begin{groupplot}[
		group style={
				group name=g, group size=2 by 2,
				horizontal sep=4em,
				vertical sep=3em,
				%x descriptions at=edge bottom,
			},
		cycle list name=samples,
		width=0.51\textwidth,
		height=0.5\linewidth,
		axis on top,
	]
	\nextgroupplot[
		title={structure factor},
		xlabel={$q\sigma$}, ylabel={$S(q)$},
		xmode=log, ymode=log, xmin=0.2, xmax=20, ymin=0.1,
		restrict x to domain=-1.8:3, unbounded coords=discard, 
		%clip mode=individual
		]
		\addplot+[clip mode=individual] file{/data/Documents/tex/svn/crystal-gel/data/fig3c_1.dat};
		\addplot+[clip mode=individual] file{/data/Documents/tex/svn/crystal-gel/data/fig3c_2.dat};
		\addplot file{/data/Documents/tex/svn/crystal-gel/data/fig3c_3.dat};
		\addplot+[clip mode=individual] file{/data/Documents/tex/svn/crystal-gel/data/fig3c_4.dat};
		\addplot+[clip mode=individual] file{/data/Documents/tex/svn/crystal-gel/data/fig3c_5.dat};
		\addplot[black, ultra thick, domain=0.1:3] {2.14605*x^(-3)} node[pos=0.92,left=0.5em] {$D=3$};
		\addplot[gray, ultra thick, domain=0.1:3] {3.98797*x^(-2.53)} node [pos=0.65, above right] {$D=2.53$};
	
	\nextgroupplot[
		title={local volume fraction},
		xlabel={\# neighbours}, ylabel={$\phi$ [\%]},
		xmin=0, ymin=0, ytick={0,10,30,50},
		extra y ticks={64, 72}, extra y tick style={grid=major}, extra y tick label={},
		legend pos=north west, legend style={font=\footnotesize},
		no markers,
		/pgfplots/error bars/y dir=both, /pgfplots/error bars/y explicit,
		]
		\addplot table[y error=err]{/data/Documents/tex/svn/crystal-gel/data/vol_err_phi_0.10_cp_0.82.scaled};
		\addplot table[y error=err]{/data/Documents/tex/svn/crystal-gel/data/vol_err_phi_0.10_cp_1.36.scaled};
		\addplot table[y error=err]{/data/Documents/tex/svn/crystal-gel/data/vol_err_phi_0.25_cp_0.38.scaled};
		\addplot table[y error=err]{/data/Documents/tex/svn/crystal-gel/data/vol_err_phi_0.25_cp_0.48.scaled};
		\addplot table[y error=err]{/data/Documents/tex/svn/crystal-gel/data/vol_err_phi_0.25_cp_0.57.scaled};
	
	
	\end{groupplot}
	%\draw(g c1r1.outer south west) -- +(\textwidth,0);
	\end{tikzpicture}
	\begin{itemize}
	\item Final stucture compatible with random gelation
	\item Glassy local $\phi$ for 12-neighboured particles
	\end{itemize}
	\alert{Except} \textcolor{red}{$\bullet$}: compact structure and higher local $\phi$. %\textcolor{ilmorange}{$\Rightarrow$} Crystal?
\end{frame}

\begin{frame}{Real space: crystal gel!}
	\tikzsetnextfilename{phdiag_arrows}%
	\begin{tikzpicture}[
			every axis/.style={
			xlabel absolute, every axis x label/.append style={anchor=base, yshift=-1em},
			%ylabel absolute, every axis y label/.append style={anchor=base, yshift=-0.5em}
			}]
		\pgfplotscreateplotcyclelist{samples}{%
		blue,mark=square*,\\%
		cyan,mark=square\\%
		red, thick, mark=*\\%
		red!65!yellow, mark=otimes\\%
		red!30!yellow, mark=oplus\\%
		red!65!black, mark=o\\%
		}%
		\begin{axis}[
			cycle list name=samples,
			width=0.51\textwidth,
			height=0.5\linewidth,
			axis on top,
			xlabel={$\phi$}, ylabel={$c_p$ (\si{\milli\gram/\gram})},
			xmin=0,xmax=0.74,
			ymin=0, ymax=1.9, ytick={0,0.5,...,2},
			mark options={solid}
			]
			\begin{scope}[/pgfplots/skip coords between index={0}{1}, /pgfplots/skip coords between index={2}{3}]%[draw=none,mark repeat=2,mark phase=2]
			\addplot table {/data/Documents/tex/svn/crystal-gel/sample0.phd};
			\addplot table {/data/Documents/tex/svn/crystal-gel/sample1.phd};
			\addplot table {/data/Documents/tex/svn/crystal-gel/sample2.phd}[->] -- (rel axis cs:0.95,0.2);
			\addplot table {/data/Documents/tex/svn/crystal-gel/sample3.phd};
			\addplot table {/data/Documents/tex/svn/crystal-gel/sample4.phd};
			\addplot table {/data/Documents/tex/svn/crystal-gel/sample5.phd}[->] -- (rel axis cs:0.95,0.7);
			\end{scope}
			\addplot[black, dashed,no marks] file {/data/Documents/tex/svn/crystal-gel/gasliquid_sg.phd};% node[diamond, fill, inner sep=2pt] at (current plot begin){};
			\addplot[black, dashed,no marks] file {/data/Documents/tex/svn/crystal-gel/gasliquid_sl.phd};
			\addplot[black, no marks] file {/data/Documents/tex/svn/crystal-gel/gasliquid_bl.phd};
			\addplot[black, no marks] file {/data/Documents/tex/svn/crystal-gel/gasliquid_bg.phd};
			\addplot[fill=lightgray!50, no marks, draw=none] file {/data/Documents/tex/svn/crystal-gel/fluidcrystal_f.phd} node[above right] at (rel axis cs:0,0) {Fluid} -- (rel axis cs:0,0) \closedcycle;
			\addplot[fill=lightgray!50, no marks, draw=none] file {/data/Documents/tex/svn/crystal-gel/fluidcrystal_x.phd} %node[above left] (x) at (rel axis cs:1,0.5) {Crystal}  
			-| (rel axis cs:1,0) \closedcycle;
			%\draw[->] (x.base) --(rel axis cs:0.98,0.01);
			\addplot[only marks, mark=triangle] file{/data/Documents/tex/svn/crystal-gel/fluid_samples.phd};
	\end{axis}
	\end{tikzpicture}
	\includegraphics[height=11\baselineskip,clip=true,trim=0.5cm 0 0 0]{/data/Documents/tex/svn/crystal-gel/nature/fig2.pdf}
	
	\bigskip
	Compact structure and higher local $\phi$ \alert{because crystallized}
\end{frame}

\appendix
\newcounter{finalframe}
\setcounter{finalframe}{\value{framenumber}}

\begin{frame}[plain]
\end{frame}

\setcounter{framenumber}{\value{finalframe}}
\end{document}

